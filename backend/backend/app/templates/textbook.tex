
\documentclass[11pt]{book}
\usepackage[margin=1in]{geometry}
\usepackage{amsmath,amsfonts,amssymb}
\usepackage{graphicx}
\usepackage{hyperref}
\usepackage{fancyhdr}
\usepackage{amsthm}
\usepackage{enumitem}
\usepackage{tocloft}

% Theorem environments
\newtheorem{definition}{Definition}[chapter]
\newtheorem{theorem}{Theorem}[chapter]
\newtheorem{lemma}{Lemma}[chapter]
\newtheorem{corollary}{Corollary}[chapter]
\newtheorem{example}{Example}[chapter]
\newtheorem{remark}{Remark}[chapter]

% Header and footer setup
\pagestyle{fancy}
\fancyhf{}
\fancyhead[LE,RO]{\thepage}
\fancyhead[LO,RE]{\leftmark}
\renewcommand{\headrulewidth}{0.4pt}

% Title page information
\title{\textbf{MATHEMATICS TEXTBOOK TEMPLATE}}
\author{Author Name}
\date{\today}

\begin{document}

% Title page
\begin{titlepage}
\centering
\vspace*{2cm}

{\Huge\bfseries MATHEMATICS TEXTBOOK TEMPLATE\par}
\vspace{1cm}
{\Large A Comprehensive Guide to Mathematical Concepts\par}
\vspace{2cm}

{\large Author Name\par}
\vspace{1cm}

{\large Department of Mathematics\par}
University Name\par
\vspace{1cm}

{\large \today\par}

\vfill

\end{titlepage}

% Copyright page
\newpage
\thispagestyle{empty}
\vspace*{\fill}
\begin{center}
Copyright \copyright\ 2024 Author Name\\
All rights reserved.
\end{center}
\vspace*{\fill}

% Table of contents
\tableofcontents
\newpage

% Preface
\chapter*{Preface}
\addcontentsline{toc}{chapter}{Preface}

This textbook provides a comprehensive introduction to mathematical concepts. The material is organized to build understanding progressively, with each chapter building upon previous knowledge.

\textbf{Prerequisites:} Basic algebra and geometry.

\textbf{How to use this book:} Each chapter contains definitions, theorems, examples, and exercises. Work through the examples carefully and attempt all exercises to reinforce your understanding.

% Part I
\part{Part I Title}

\chapter{Chapter 1 Title}
\section{Section 1.1 Title}
\begin{definition}
This is a placeholder definition. Replace with your actual definition.
\end{definition}

\begin{example}
This is a placeholder example. Replace with your actual example.
\end{example}

\section{Section 1.2 Title}
\begin{definition}
This is another placeholder definition. Replace with your actual definition.
\end{definition}

\begin{theorem}[Theorem Name]
This is a placeholder theorem. Replace with your actual theorem statement.
\end{theorem}

\begin{proof}
This is a placeholder proof. Replace with your actual proof.
\end{proof}

\section{Exercises}
\begin{enumerate}
\item This is a placeholder exercise. Replace with your actual exercise.
\item This is another placeholder exercise. Replace with your actual exercise.
\item This is a third placeholder exercise. Replace with your actual exercise.
\end{enumerate}

\chapter{Chapter 2 Title}
\section{Section 2.1 Title}
\begin{definition}
This is a placeholder definition. Replace with your actual definition.
\end{definition}

\begin{lemma}[Lemma Name]
This is a placeholder lemma. Replace with your actual lemma statement.
\end{lemma}

\section{Section 2.2 Title}
\begin{corollary}[Corollary Name]
This is a placeholder corollary. Replace with your actual corollary statement.
\end{corollary}

\section{Exercises}
\begin{enumerate}
\item This is a placeholder exercise. Replace with your actual exercise.
\item This is another placeholder exercise. Replace with your actual exercise.
\item This is a third placeholder exercise. Replace with your actual exercise.
\end{enumerate}

% Part II
\part{Part II Title}

\chapter{Chapter 3 Title}
\section{Section 3.1 Title}
\begin{definition}
This is a placeholder definition. Replace with your actual definition.
\end{definition}

\begin{theorem}[Theorem Name]
This is a placeholder theorem. Replace with your actual theorem statement.
\end{theorem}

\section{Section 3.2 Title}
\begin{remark}
This is a placeholder remark. Replace with your actual remark.
\end{remark}

\section{Exercises}
\begin{enumerate}
\item This is a placeholder exercise. Replace with your actual exercise.
\item This is another placeholder exercise. Replace with your actual exercise.
\item This is a third placeholder exercise. Replace with your actual exercise.
\end{enumerate}

% Part III
\part{Part III Title}

\chapter{Chapter 4 Title}
\section{Section 4.1 Title}
\begin{definition}
This is a placeholder definition. Replace with your actual definition.
\end{definition}

\begin{theorem}
This is a placeholder theorem. Replace with your actual theorem statement.
\end{theorem}

\section{Section 4.2 Title}
\begin{example}
This is a placeholder example. Replace with your actual example.
\end{example}

\section{Exercises}
\begin{enumerate}
\item This is a placeholder exercise. Replace with your actual exercise.
\item This is another placeholder exercise. Replace with your actual exercise.
\item This is a third placeholder exercise. Replace with your actual exercise.
\end{enumerate}

% Bibliography
\chapter*{Bibliography}
\addcontentsline{toc}{chapter}{Bibliography}

\begin{thebibliography}{9}
\bibitem{stewart}
Stewart, James. \emph{Calculus: Early Transcendentals}. 8th ed. Cengage Learning, 2016.

\bibitem{spivak}
Spivak, Michael. \emph{Calculus}. 4th ed. Publish or Perish, 2008.

\bibitem{rudin}
Rudin, Walter. \emph{Principles of Mathematical Analysis}. 3rd ed. McGraw-Hill, 1976.
\end{thebibliography}

\end{document}
